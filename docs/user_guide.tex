%% set for XeTex + Latex
%\documentclass[letterpaper,landscape]{article}
\documentclass[letterpaper]{article}
\usepackage[T1]{fontenc} 
\usepackage{type1cm}
\usepackage[margin=3cm]{geometry}
\usepackage{makeidx}
%\usepackage{2in1, lscape}
\usepackage{lastpage}
\usepackage{fancyhdr}

\usepackage{fontspec}
\usepackage{xunicode}
\usepackage{xltxtra}
\setmainfont[Mapping=tex-text]{Palatino Linotype}
\setmonofont{Monaco}

\pagestyle{fancy}
\fancyfoot[C]{{\thepage} of \pageref{LastPage}}
\setlength\headheight{12pt}		% sets page header height
\setlength\headsep{12bp}		% changes height of title to the line above it
\makeindex
%\renewcommand{\labelitem}{$\diamond$}
%\renewcommand{\thefootnote{\fnsymbol{footnote}}}
%\usepackage{indentfirst}
%\setlength{\parskip}{3mm}
%\setlength{\parindent}{3mm}
\setlength{\parindent}{0mm}

\def\xbe{Haiku}
\def\xlin{Linux}
\def\xmswin{Microsoft Windows}
\def\xboxsize{4.5in}

\def\xapp{Another Linux File Commander}
\def\xauthor{Stu George}
\def\xdeveloper{Bloody Cactus Development Labs}
\def\xlicense{GNU GPL version 2}

\begin{document}

\title{\xapp}
\author{\xauthor\\\xdeveloper}
\date{\today}
\maketitle

%\begin{abstract}
%\xapp End User documentation
%\end{abstract}

\vfill \eject

\tableofcontents

\vfill \eject
% after we have TOC, change paragraph skip size
\setlength{\parskip}{3mm}
\section{Introduction}
\subsection{Welcome}
Thank you for using Another Linux File Commander (herein noted as \textsl{ALFC}).

\begin{flushright}
-- \xauthor
\end{flushright}

\vfill \eject
\subsection{What is ALFC?}
\textsl{ALFC} is what is known as a \textsl{clone} of the famous \textsl{Norton Commander} 
file management program.

The overall goal is to empower the end user as far as file management goes.

This is done by using what is known as a dual pane file manager, with files listed on the left
and the right, usually in a source and target configuration. For example, you can copy files from left to right or right to left.

A lot of \textsl{ALFC}'s power comes from its scriptability which is drive by the Lua scripting language.

\subsection{Assumptions}
This user guide makes the assumption in its text and examples that you have not altered the
basic out-of-the-box configuration.

We do not assume one particular platform, as \textsl{ALFC} can be run on either \xmswin\ or \xlin, and we will endevour to mark any platform specific examples. \xbe\ is a bit different but close enough to be posix compatible, it just has a different directory layout.

\xlin\ can be used interchangably with most other posix operating systems such as FreeBSD

\vfill \eject
\section{Configuration}
\index{Configuration}
Configuration is achieved through \textsl{ALFC}'s main option file as well as its lua scripting files.

\subsection{Configuration Files}
\index{Configuration!Config Files}
\textsl{ALFC} uses one main configuration file for basic options

Newer versions of \xmswin\ from XP and onwards;
\begin{flushleft}
\framebox{\parbox{\xboxsize}{WINDOWS XP,VISTA,7\\\texttt{\$USERPROFILE\textbackslash \$USERNAME\textbackslash .alfc
\\eg: C:\textbackslash Documents and Settings\textbackslash username\textbackslash .alfc}}}
\end{flushleft}

Older versions of \xmswin such as the Windows 98 and Windows ME family;
\begin{flushleft}
\framebox{\parbox{\xboxsize}{WINDOWS 98\\\texttt{\$USERPROFILE\textbackslash .alfc
\\eg: C:\textbackslash Windows\textbackslash Profiles\textbackslash username\textbackslash .alfc}}}
\end{flushleft}

Posix and Unix systems store it in the home directory;
\begin{flushleft}
\framebox{\parbox{\xboxsize}{LINUX\\\texttt{\$(HOME)/.alfc}}}
\end{flushleft}

BeOS / Haiku is a little different;
\begin{flushleft}
\framebox{\parbox{\xboxsize}{HAIKU\\\texttt{\$(HOME)/.alfc
\\eg: /boot/home}}}
\end{flushleft}


\section{File Management}

\subsection{File Attributes}
A file or directory can have several attributes attached to it. Under \xmswin, they can be Archive, Read-Only, 
Compressed, System and Hidden. Under \xlin, they can be Executable, Readable, Writeable, Hidden.

\subsubsection{Executable}
Applies to \xlin only.

\subsubsection{Read Only}
Applies to both \xlin\ and \xmswin. 

\subsubsection{Hidden Files}
\begin{center}
\framebox{\parbox{\xboxsize}{WINDOWS\\On \xmswin, any file that starts with a dot eg ".netbeans" will have
its attribute shown as hidden, even when it does not have a hidden 
attribute.}}
\end{center}

\subsubsection{Writable}
Applies to \xlin\ only.

\subsubsection{System}
Applies to \xmswin\ only.

\subsubsection{Compressed}
Applies to \xmswin\ only.

\subsubsection{Archive}
Applies to \xmswin\ only.

\subsection{Tags, Globs and Filters}
\subsubsection{Globs}
\index{Tagging!glob}The term \textsl{Glob} indicates a simple, limited form of pattern matching.
\begin{center}
\framebox{\parbox{\xboxsize}{GLOBBING\\\texttt{'*.doc' refers to all files ending in '.doc'. If you wanted}}}
\end{center}

Globbing has two special characters, the ? and the *;
 
\begin{center}
    \begin{tabular}{ | l | l | l |}
    \hline
    Character & Description & Example \\ \hline
    ? & Single character & ?at = Cat, Bat, Hat\\ \hline
    * & Multi character Wildcaard & d*n = don, down, doooooooown, dominican \\ \hline
    \hline
    \end{tabular}
\end{center}


\subsubsection{Filters}
\index{Tagging!filter}
Filters are a more complex form of pattern matching, known as \index{Tagging!regular expressions}regular expressions.

In filters, the special characters of ?, * and + mean different things compared to the ? and * used in globbing.



\subsubsection{Tags with Globs and Filters}
\index{Tagging!tags}


\vfill \eject
\section{Credits}
\textsl{ALFC} was designed and developed by \xdeveloper\ through fits and spurts of time.

Documentation compiled on \today

Thank you for reading.

\section{Licensing}
\index{Licensing}
The binary application, source code and accompanying documention/manuals are all
available under the \index{GNU GPL}GNU \index{GPL}GPL v2 license.

\subsection{Trademarks}
\index{Trademarks}
Windows\textregistered\ is a registered trademark of Microsoft Corporation in the United States and other countries.

Linux\textregistered\ is the registered trademark of Linus Torvalds in the U.S. and other countries.

\setlength{\parindent}{0mm}

\vfill \eject
\printindex

\end{document}

